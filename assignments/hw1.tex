\documentclass[letterpaper]{article}
\usepackage[margin=1in]{geometry}
\usepackage{amsmath, amssymb}
\usepackage{titlesec}
\usepackage{enumerate}
\usepackage{fancyhdr}

\pagestyle{fancy}
\fancyhead[L]{STAT 543 HW1}
\fancyhead[R]{Xin Yin}
\titleformat{\section}{\em\Large} {$\dagger$ \thesection .}{10pt}{}

\newcommand{\bx}{\mathbf{x}}
\newcommand{\bX}{\mathbf{X}}
\newcommand{\Tx}{\mathbf{T(x)}}
\newcommand{\TX}{\mathbf{T(X)}}

\begin{document}
    \section{}
        Given that $X_1, X_2, X_3$ are random sample from Ber$(\theta)$, we have that
        \[
        f(x_i|\theta) = \theta^{x_i} (1-\theta)^{1-x_i}, \quad i = 1, 2, 3
        \]
        and that,
        \[
        f(\bx|\theta) = \prod_i \theta^{x_i} (1-\theta)^{1-x_i} = \theta^{\sum_i x_i} (1-\theta)^{\sum_i (1-x_i)}.
        \]

        Now, by definition, if $T = X_1 + 2 X_2 + X_3$ is a sufficient statistic for $\theta$, $P(\bX = \bx|\TX=\Tx)$ should be independent of $\theta$. However, we can show that,
        \begin{align*}
        P(\bX = (0, 1, 0)|\TX = 2) & = \frac{P(\bX = (0, 1, 0)}{P(\TX = 2)} \\
        & = \frac{P(\bX = (0, 1, 0)}{P(\bX \in \{(1, 0, 1), (0, 1, 0)\})} \\
        & = \frac{\theta (1-\theta)^2}{\theta(1-\theta)^2 + \theta^2(1-\theta)} \\
        & = \frac{(1-\theta)^2}{(1-\theta)^2 + \theta(1-\theta)},
        \end{align*}
        which is not independent of $\theta$. Thus, $\TX = X_1 + 2X_2 + X_3$ is not a sufficient statistic of $\theta$.

    \section{}
        Given the pdf,
        \[
        f_{X_i}(x|\theta) = \begin{cases}
            e^{i\theta-x} & x \ge i\theta \\
            0   & x < i\theta
        \end{cases}
        \]
        We can write down the likelihood as, given the independence of $X_1, \dots, X_n$,
        \[
        f(\bx|\theta) = \prod_{i=1}^n e^{i\theta-x_i} I(x_i \ge i\theta) 
        = e^{\theta \sum_{i=1}^n i} I(\min_{i}(X_i/i) \ge \theta) \cdot e^{-\sum_{i=1}^n x_i},
        \]
        which factorized into $g(\Tx|\theta) = e^{\theta \sum_{i=1}^n i} I(\min_{i}(X_i/i) \ge \theta)$ where $T = \min_{i} (X_i/i)$, and $g(x) = e^{-\sum_{i=1}^n x_i}$. Therefore, $T = \min_{i} (X_i/i)$ is a sufficient statistic for $\theta$.
    \section{}
        The likelihood for pdf $f(x|\mu, \sigma) = \frac{1}{\sigma} e^{-(x-\mu)/\sigma}, \mu < x < \infty$ is,
        \[
        f(\bx|\theta) = \prod_{i=1}^n \frac{1}{\sigma} e^{-(x_i - \mu)/\sigma} I(x_i > \mu) = \left(\frac{e^{\mu/\sigma}}{\sigma}\right)^n e^{-\cfrac{\sum_{i=1}^n x_i}{\sigma}} I(X_{(1)} > \mu) \cdot 1,
        \]
        which can be factorized into $g(\Tx|\theta) = \left(\frac{e^{\mu/\sigma}}{\sigma}\right)^n e^{-\cfrac{\sum_{i=1}^n x_i}{\sigma}} I(X_{(1)} > \mu)$ and $h(x) = 1$. 
        So, $(X_{(1)}, \sum_{i=1}^n X_i)$ is a sufficient statistic for $(\mu, \sigma)$.
    \section{}
        Given that $X_1, \dots, X_n$ are random samples from Uniform$(\theta, 2\theta)$ distribution, the likelihood can be written down as,
        \[
        f(\bx|\theta) = \prod_{i=1}^n \frac{1}{\theta} I(\theta < x < 2\theta) = \frac{1}{\theta^n} I(\frac{X_{(n)}}{2} < \theta < X_{(1)}) \cdot 1,
        \]
        which factorize into $g(\Tx|\theta) = \frac{1}{\theta^n} I(\frac{X_{(n)}}{2} < \theta < X_{(1)})$, and $h(x) = 1$. Therefore, $(X_(1), X_(n))$ is a sufficient statistic for $\theta$.
        \section{}
        Given that $X_1, \dots, X_n$ are random samples from discrete uniform distribution on $1, 2, \dots, \theta$, we know that,
        \begin{eqnarray*}
        f(x|\theta) = \frac{1}{\theta}I(x \in \mathcal{N}_\theta),
        F(X|\theta) = \frac{x}{\theta}I(x \in \mathcal{N}_\theta),
        \end{eqnarray*}
        where $\mathcal{N}_\theta = \{1, 2, \dots, \theta\}$.

        Then, the distribution for $T = X_{(n)}$ is,
        \begin{align*}
        q(\Tx|\theta) & = \frac{n!}{(n-1)!} f_{\Tx}(x|\theta) [F_{\Tx}(x|\theta)]^{n-1} [1-F(x|\theta)]^0 \\
        & = \frac{n}{\theta} \left(\frac{\Tx}{\theta}\right)^{n-1} I\left(\Tx \in \mathcal{N}_\theta\right) \\
        & = \frac{n}{\theta^n} \Tx^{n-1} I\left(\Tx \in \mathcal{N}_\theta\right)
        \end{align*}
        And the likelihood for the joint pmf is,
        \[
        p(\bx|\theta) = \prod_{i=1}^n \frac{1}{\theta} I(x_i \in \mathcal{N}_\theta) = \frac{1}{\theta^n} I(\Tx \in \mathcal{N}_\theta) \left(\prod_{i=1}^n I(x_i \in \mathcal{N})\right),
        \]
        where $\Tx = X_{(n)}$,  and $\mathcal{N} = \{1, 2, \dots\}$.

        By definition, $\Tx$ will be a sufficient statistic if $P(\bX = \bx|\TX = \Tx)$ is independent of $\theta$. And,
        \[
        P(\bX = \bx|\TX = \Tx) = \frac{p(\bx|\theta)}{q(\Tx|\theta)} = \cfrac{\frac{1}{\theta^n} I(\Tx \in \mathcal{N}_\theta) \left(\prod_{i=1}^n I(x_i \in \mathcal{N})\right)}{\frac{n}{\theta^n} \Tx^{n-1} I\left(\Tx \in \mathcal{N}_\theta\right)} = \cfrac{\prod_{i=1}^n I(x_i \in \mathcal{N})}{n \Tx^{n-1}},
        \]
        is independent of $\theta$. Therefore, $\Tx = X_{(n)}$ is a sufficient statistic for $\theta$.
        \section{}
        \begin{enumerate}[(a)]
        \item 
        If $\TX = X_1 + X_2$ is a sufficient statistic, by definition, the conditional probability $P(\bX = \bx|\TX = \Tx)$ should be independent of $p$. However,
        \begin{align*}
        P(\bX = (0, 1)|\TX = 1) & = \frac{P(X_1 = 0, X_2 = 1)}{P(\TX = 1)} \\
        & = \frac{(1-p)5p}{(1-p)5p + p(1-5p)} = \frac{5(1-p)}{5(1-p) + (1-5p)},
        \end{align*}
        is not independent of $p$. Thus, $\TX = X_1 + X_2$ is not a sufficient statistic of $p$.

        \item
        Since $\TX = X_1 + 3X_2$ has a one-to-one mapping from $(X_1, X_2)$ to $X_1 + 3X_2$, we have that $f_{X_1+3X_2}(\Tx|p) = f(x_1, x_2|p)$. 

        Then, $\frac{p(\bx|p)}{q(\Tx|\theta)}$ will remain as a constant and thus independent of $p$. So, $X_1 + 3X_2$ is a sufficient statistic for $p$.
        \end{enumerate}
\end{document}
